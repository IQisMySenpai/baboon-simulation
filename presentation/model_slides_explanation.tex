\documentclass[
    aspectratio=169,
    10pt,
]{beamer}
\usepackage{graphicx} % Required for inserting images

\usepackage[backend=biber]{biblatex}
\addbibresource{references.bib}
\begin{filecontents}{references.bib}
@article{StrandburgPeshkin2015,
  title={Shared decision-making drives collective movement in wild baboons},
  author={Strandburg-Peshkin, Ariana and Farine, Damien R. and Couzin, Iain D. and Crofoot, Margaret C.},
  journal={Science},
  volume={348},
  number={6241},
  pages={1358--1361},
  year={2015},
  publisher={American Association for the Advancement of Science},
  doi={10.1126/science.aaa5099},
  url={https://www.science.org/doi/10.1126/science.aaa5099}
}

@book{oksendal2003stochastic,
  author = {{\O}ksendal, Bernt},
  day = 21,
  edition = {6th},
  howpublished = {Paperback},
  isbn = {3540047581},
  month = jan,
  publisher = {Springer},
  timestamp = {2019-08-24T00:34:38.000+0200},
  title = {{Stochastic Differential Equations: An Introduction with Applications (Universitext)}},
  year = 2014,
  url = https://doi.org/10.1007/978-3-642-14394-6,
}

@misc{chen2019neuralordinarydifferentialequations,
      title={Neural Ordinary Differential Equations}, 
      author={Ricky T. Q. Chen and Yulia Rubanova and Jesse Bettencourt and David Duvenaud},
      year={2019},
      eprint={1806.07366},
      archivePrefix={arXiv},
      primaryClass={cs.LG},
      url={https://arxiv.org/abs/1806.07366}, 
}

@misc{kidger2021neuralsdesinfinitedimensionalgans,
      title={Neural SDEs as Infinite-Dimensional GANs}, 
      author={Patrick Kidger and James Foster and Xuechen Li and Harald Oberhauser and Terry Lyons},
      year={2021},
      eprint={2102.03657},
      archivePrefix={arXiv},
      primaryClass={cs.LG},
      url={https://arxiv.org/abs/2102.03657}, 
}
\end{filecontents}

\usepackage{amsmath, amssymb, amsthm}

\title{Baboons presentation}
\author{Pablo Cuesta Sierra}
\date{May 2025}

% Looks
\setbeamercolor{frametitle}{fg=black}
\setbeamertemplate{navigation symbols}{}
\setbeamersize{text margin left=0.9cm, text margin right=0.9cm}

\usepackage{fontspec}
\setsansfont{Arial}
\setmainfont{Arial}
\usepackage{unicode-math}
\unimathsetup{math-style=TeX}
\setmathfont{Latin Modern Math}


% Image on left bottom
\usepackage{tikz}
\usepackage{graphicx}
\addtobeamertemplate{footline}{}{%
  \begin{tikzpicture}[remember picture,overlay]
    \node[anchor=south west, xshift=0.7cm, yshift=-0.1cm] at (current page.south west) {
      \includegraphics[height=1cm]{ETH Zurich New}
    };
  \end{tikzpicture}
}

% Frame title
\setbeamerfont{frametitle}{series=\normalfont}
\makeatletter
\setbeamertemplate{frametitle}{%
  \nointerlineskip%
  \vskip1.5ex%
  \makebox[\dimexpr\paperwidth-1.4cm][l]{\hspace*{0cm}\insertframetitle}%
  \vskip-2ex%
}
\makeatother

% Commands
\newcommand{\independent}{\perp\!\!\!\!\perp}


\begin{document}

\begin{frame}
\frametitle{Stochastic Differential Equations (SDEs): basic idea}


% \begin{columns}
    % \begin{column}{0.5\textwidth}
       SDEs describe dynamical systems with some source of randomness. \pause

       They generalize ordinary differential equations (ODEs)
        \[
           \frac{dX_t}{dt} = f(X_t, t)
        \]
        by incorporating a noise term 
        \[
           \frac{dX_t}{dt} = \pause f(t, X_t) + \text{``noise}_t\text{''}(\omega).
        \]
    % \end{column}
    % \pause
    % \begin{column}{0.5\textwidth}
    \pause
        In practice,\[
            dX_t = f(t, X_t)\, dt + dW_t,
        \]
        where \(
            W_t
        \) is a Brownian motion.
%     \end{column}
% \end{columns}

\end{frame}

\begin{frame}
\frametitle{SDEs and simulation}

\vspace{0.7cm}

\textbf{General Form:} 
\[
    X_0 = x \in \mathbb{R}^N
\]
\[
    \underbrace{dX_t}_{N\times 1} = \underbrace{f(t, X_t)}_{N\times 1}\, dt + \underbrace{g(t, X_t)}_{N\times B}\, \underbrace{dW_t}_{B\times 1},
\]
\begin{itemize}
    \item \( f(X_t, t) \) is the \textbf{drift coefficient}.
    \item \( g(X_t, t) \) is the \textbf{diffusion coefficient}.
\end{itemize}
\pause
\bigskip
\textbf{Key Features:}
\begin{itemize}
    \item Models systems with inherent uncertainty (e.g., financial markets, image generation, time series prediction, animal or particle movement).
    \pause
    \item Numerous methods for numerical solutions: e.g., Euler-Maruyama method:\[
        \hat X_{t+\Delta t} = \pause \hat X_t + \pause f(t, \hat X_t)\Delta t  \pause + g(t, \hat X_t)\underbrace{\Delta W_t}_{\sim \mathcal N(0, t)}
    \]
\end{itemize}

\end{frame}


\begin{frame}[fragile]{Modeling Baboon Movement with SDEs}
    
        
    \(B\): number of baboons, \(x = x^{1:B} = (x^1, \dots, x^B)\): paths of the baboons,
    \begin{align*}
        x^i:[0, T] &\to \mathbb R^2\\
        t&\mapsto x^i_t.
    \end{align*} \pause
    % \(x^i_t\) is the position of \(i\)-th baboon at time \(t\).
    \(    x^i    \) follows the SDE
    
    \begin{equation}
        dx^i_t = f^i(t, x_{0:t}, \omega) dt + \alt<3->{\sigma}{g^i(t, x_{0:t}, \omega)} dW^i_t(\omega).
    \end{equation}
    \pause
    In practice:
    \begin{verbatim}
    x[t+1]  =  x[t]   +   drift(x[:t]) * dt   +   sigma * normal_sample(0,dt)
    \end{verbatim}
    \pause
    \vspace{0.2cm}
    \begin{center} 
        \large
        \emph{The baboon strategy is fully defined by} {\texttt{drift()}}
    \end{center}
\end{frame}
\begin{frame}[fragile]{Modeling Baboon Movement with SDEs: advantages}
    \begin{itemize}
        \item Generalizes usual approaches: \(x_{t+1} = x_t + \text{step}\) (we don't lose generality)
        \pause
        \item Simple code \pause (we only care about defining \texttt{drift} properly).
        \pause 
        \item We can reuse well-understood numerical methods to obtain simulations.
        \pause 
        \item We can \emph{learn} the drift coefficients from data: Neural (O/C/S)DEs \cite{chen2019neuralordinarydifferentialequations, kidger2021neuralsdesinfinitedimensionalgans}
        \pause
        \begin{figure}
            \centering
            \includegraphics[width=0.7\linewidth]{neuralSDEs_diagram.pdf}
        \end{figure}
        \pause
        \item We could model other environments if we modify our equation driver:
        \[
            dX_t = f(...) \cdot dt + g(...) \cdot 
            \alt<8->{\underbrace{d\left(\ \mu(t)\, t + W_t\ \right)}_{\makebox[0pt]{\footnotesize\text{driven by Brownian motion with drift}}}}{\underbrace{dW_t}_{\makebox[0pt]{\footnotesize\text{driven by Brownian motion}}}}
        \]
    \end{itemize}
\end{frame}

\begin{frame}{Future work}
    \begin{itemize}
        \pause
        \item Explore the possibility of training a Neural SDE model with the original GPS data 
        \begin{figure}
            \centering
            \includegraphics[width=0.7\linewidth]{neuralSDEs_diagram.pdf}
        \end{figure}
        \pause
        \item Explore different equation drivers \(Y_t\) alternative to plain Brownian motion to model the movement of (other) animals moving in different mediums (e.g. a fluid, windy conditions, etc.)
        \[
            dX_t = f(...) dt + g(...) dY_t
        \]
    \end{itemize}
\end{frame}

\begin{frame}{References}

\nocite{*}
\printbibliography
    
\end{frame}

% Frames just for pictures
\setbeamertemplate{footline}{}

\begin{frame}
\Huge
\[  
    dX_t = f(...) dt + g(...) dW_t
\]
    
\end{frame}
\begin{frame}
\LARGE
\[  
    dX_t = f(...) dt + g(...) d\left(
        \underbrace{5\sin(t)\begin{bmatrix}
            \sin(t / K) \\
            \cos(t / K)
        \end{bmatrix}}_{\mu(t)} t + W_t
    \right)
\]
    
\end{frame}



\end{document}



% WORKFLOW for google slides:
% 1. Compile this 
% 2. Convert to PNG higest resolution possible in https://www.ilovepdf.com/pdf_to_jpg
% 3. Upload to google slides :/